\section{Contest of Skills}\label{sec:contest}
Situations arise, when your character needs to test their skills and prove their worth, this is done via a \textit{Contest of Skills}.
These contests come in three different variants: \textit{Passive}, \textit{Active}, and \textit{Tournament}.

\subsection{Passive Contest}
A passive contest arises when the character needs to overcome some static obstacle. 
This could be things like walking a tightrope, picking a lock, leaping over a gap, etc.
The GM silently decides the level of difficulty by announcing a skill $\pm$ a modifier that needs to be rolled.

\begin{example} 
    Astrid is running from her pursuers across the rooftops of ancient Rome, she needs to clear a particularly wide gap between the buildings. 
    Astrid says \textit{``I'm jumping the gap''}, to which the GM replies \textit{``Roll Athletics -2''}. 
    Astrid's Athletics skill is at 11, at -2, she must roll 9 or less to succeed. She rolls an 8 and gracefully leaps between the buildings.
\end{example}

\subsection{Active Contest}
An active contest is when the character is up against someone or something that actively works against them. 
Be it trying to pick-pocket a guard, an intense sword-fight, or playing chess against an opponent; all of this providing an \textit{active} resistance.

In an active contest, both parties roll against their most applicable skill, the outcome is judged as follows:

\begin{center}
    \begin{tabular}{c|c|c}
    \textbf{Character A} & \textbf{Character B} & \textbf{Winner}\\\hline
    Fail    & Fail    & The one failing by least \\
    Fail    & Success & Character B \\
    Success & Fail    & Character A \\
    Success & Success & The one succeeding by most
    \end{tabular}
\end{center}
If no one fails or succeeds more than the opponent, re-roll.

\begin{example} 
    Tom wants to sneak past a guard, Tom's \textit{Stealth} is at 9, he rolls 11, failing by 2 degrees. 
    The guard's \textit{Perception} is at 10, but rolls 13, failing by 3 degrees. 
    Tom failed less than the guard, thus successfully sneaking past them.
\end{example}

\subsection{Tournaments}
Tournaments are \textit{longer} contests of skills.
Tournaments are divided into rounds, whoever wins the most rounds wins the tournament.
Each individual round is a regular \textit{active contest}.

It's up to the GM to decide the number of rounds in a given contest.