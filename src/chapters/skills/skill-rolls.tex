\section{Skill Rolls}
The player rolls three six-sided dice (3d6) against their skill of choice, if the outcome is $\leq$ the skill level, it's a success, if not, it's a fail.
Rolling a 3, is an automatic success (regardless of level), and rolling an 18 is an automatic fail (also regardless of level).

\begin{example} 
    Alice wants to pick a lock. 
    Her \textit{Thievery} skill is at 13. 
    She rolls 3d6, and rolls a 1, a 5, and a 3, which is 9 in total. 
    This means she successfully picks the lock.
\end{example}

\begin{figure}
\centering
\begin{tabular}{r | c | c | l}
\textbf{Category} & \textbf{Skill}   & \textbf{Traits} & \textbf{Description} \\\hline
    Mental    & Academics        & $Int+Int$ & Use vast knowledge of a certain field.        \\
              & Investigation    & $Int+Wis$ & Searching for things and information.         \\
              & Magic            & $Int+Wis$ & Perform fantastical feats.                    \\
              & Perception       & $Int+Wis$ & The passive ability to spot things.           \\
              & Survival         & $Con+Int$ & Surviving in certain environments.            \\
              & Thievery         & $Dex+Int$ & Performing certain larcenous activities.      \\\hline
    Physical  & Athletics        & $Agi+Con$ & Perform a certain task that requires stamina. \\
              & Fighting (Heavy) & $Agi+Str$ & The ability to fight with heavy weapons.      \\
              & Fighting (Light) & $Agi+Dex$ & The ability to fight with light weapons.      \\
              & Physique         & $Con+Str$ & Any action that requires strength.            \\
              & Stealth          & $Agi+Wis$ & Sneaking around and acting unseen.            \\\hline
    Social    & Contacts         & $Cha+Int$ & Make and use connections with people.         \\
              & Insight          & $Cha+Wis$ & Sensing and social cues and motivation        \\
              & Nursing          & $Cha+Con$ & Taking care of people and non-magic healing.  \\
              & Persuasion       & $Cha+Cha$ & Manipulating people.                          \\\hline
    Technical & Crafting         & $Dex+Int$ & Making things with your hands.                \\
              & Shooting         & $Agi+Dex$ & Shooting or throwing objects.                 \\
              & Vehicle (Air)    & $Dex+Int$ & Operating motorised air vehicles.             \\
              & Vehicle (Land)   & $Dex+Wis$ & Operating motorised land vehicles.            \\
              & Vehicle (Sea)    & $Dex+Int$ & Operating motorised sea vehicles.             \\
\end{tabular}
\caption{Skill Listing}
\label{fig:skills}
\end{figure}

\subsection{Skill Modifiers \& Challenge Ratings}
A skill modifier is something which modifies a skill.
These modifiers always come in the form $Skill \pm n$, where $n$ is by how much a given skill is modified.

Skill modifiers can be applied in a variety of situations and almost always stack.
Modifiers can come from your character's specialisations (see Section~\ref{sec:specialisation}), their exploits (see Chapter~\ref{chap:exploits}), the difficulty of a given challenge, fatigue (see Section~\ref{sec:fatigue-points}), or various status effects.

Pushing a boulder could be $\mathit{Physique} - 4$, which means it \textit{subtracts} 4 from your character's Physique skill ($\pm$ any other modifiers) \textbf{before} the roll is performed, thus making it more difficult to succeed.

\paragraph{Note to players coming from other systems} The skill system may be reversed from what you're used to, and may require some re-adjustment time because in Siren you want to be aiming for a lower roll.

\paragraph{Note to GMs coming from other systems} Siren doesn't really use \textit{advantages} or \textit{disadvantages} the same way some other systems do. What you would normally do instead, is just add more pluses or minuses to a given skill's challenge rating and call it a day.
However, if you \textit{really} want that feeling of rolling extra dice and picking out the ones you need you can do so as well: To roll with advantage, roll $5d6$ and discard the highest two. To roll with disadvantage, roll $5d6$ and discard the lowest two.

\newpage
