\section{Cooperative Skills}
It sometimes happens that two or more characters will need to help each other in performing a difficult task.
For example, moving a large boulder might be too heavy for just one person, but if the whole team helps out, they may have enough combined \textit{Strength} to pull it off.

Cooperative skill rolls are facilitated by adding half of each helper's relevant skill (rounded down) to your own skill.

\begin{note} 
    The added skill does not have to be the same as the main active one. 
    After all, boarding a door does not require the same skills as holding it shut, but both of them can be used cooperatively to achieve a common goal.
\end{note}

\begin{example}
    Eloise is on stage about to perform a country song on her guitar.
    Her \textit{Persuasion} skill is 11, but it's a tough crowd that gives her a $-6$ modifier, leaving her with an effective \textit{Persuasion} of 5.
    Luckily, George is here to help!
    His excellent gun-slinging skills would be a useful display of showmanship.
    George has a \textit{Shooting} skill at 12, letting him add 6 to Eloise's effective skill for the challenge, bringing her up to a total of 11.
    Finally, Terry adds his own attempt to dance to George's revolver-spinning and Eloise's guitar-playing. 
    Terry's \textit{Athletics} Skill is 6 allowing him to add 3 Bringing Eloise's total effective skill for the group performance up to 14.
\end{example}