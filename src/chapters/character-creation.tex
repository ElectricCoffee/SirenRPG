\chapter{Character Creation}
Character creation is perhaps one of the most important aspects of an RPG.
How you design your character impacts how the game is going to be played.
This chapter is meant as a guide to help you through creating your character in Siren.

\section{Use Your Imagination}
Make your character yours.
Think of a back-story for them, give them a reason to exist in the world you're playing in.
If you are unsure about what works in the setting, ask your GM for help; after all, they're the one curating the game.
Work with the GM and the other players to create a character that best works for your game.

\paragraph{Note} The rest of the guide below is meant as a loose guide.
If the GM decides, any and all of the rules below can be changed to suit your particular game.

\section{The Character Sheet}
The character sheet comes in two pages.
Page 1 is the general stats page, which contains your traits, skills, and exploits.
Page 2 is the weapons and equipment page, which keeps track of your weapons, money, and loot.

\paragraph{Note} If you can't fit all your character's info, you can write them on the back of the character sheet!

\section{Rolling a Random Character}
\subsection{Traits}
If you wish to roll your character's traits, you can do so.
\begin{enumerate}
\item Roll 2d6 and discard the lowest die once for every trait (Agi, Cha, Con, Dex, Int, Str, Wis).
\item \textbf{HP:} $Con \times 2$
\item \textbf{FP:} $Con + Str$
\item \textbf{MP:} $Int \times Wis$
\item \textbf{Speed:} $(Agi + Con) / 2$, rounded down
\end{enumerate}

\subsection{Skills}
Calculate your character's skills based on their traits as shown in the skill listing on your character sheet.
You may choose three skills your character is proficient in.
This also requires you to think of something they can specialise in, if possible.
See chapter \ref{chap:skills} to see what exactly that means.

\subsection{Exploits}
You may come up with up to two different exploits during character creation (you have the option to add more as your character progresses).
Exploits are entirely player-defined, and are a way of personalising your character to fit your playstyle.
You can read more about exploits in chapter \ref{chap:exploits}.

\subsection{Weapon Skills}
The weapon skills are separate from the general skills in that your character can specialise in any weapon under the sun if they so desire.
For this reason, you need to fill out the weapon stats for the given weapon yourself.
To help you out, we've included a weapon list in appendix \ref{sec:weapons}, which provides a list of weapons and their relative stats.

\paragraph{Note} You cannot use a weapon skill unless you're also carrying a weapon.

\subsection{Equipment}
The equipment section lists all the stuff you're carrying.
This is everything from armour and clothes, to loot, to weapons.

\subsection{Money}
Each box in the money section is for the currency carrying, and the lines next to the boxes are meant for the amount of said currency.
So if you're playing an international spy, the boxes could hold \textit{USD}, \textit{GBP}, \textit{EUR}.
And if you're playing in a fantasy setting, they could hold \textit{Gold}, \textit{Silver}, \textit{Bronze}.
