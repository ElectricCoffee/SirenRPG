\section{Skills}
The skill-listing is flexible, and is designed as such.
If you're writing a campaign set in the Wild West, you probably don't need any of the \textit{Vehicle} skills.
Playing in a setting that has several different incompatible magic systems? Then it may be a good idea to add a skill for each.

\subsection{Trait Pairs}
A thing to keep in mind when designing new skills is what pair of traits the skill should have.
It is easy to think that you can just slap any two traits onto a skill and call it a day, but the problem might be a bit subtler than that.
Each trait represents an approach to solving a problem, and every skill relies on one or more such approaches in tandem.

Does it need careful consideration? Then you probably want \textit{Int} in there.

Does it rely on common sense or intuition? Then \textit{Wis} is a good bet.
This is actually the primary reason why \textit{Vehicle (Land)} is \textit{Wis}-based and not \textit{Int}-based; most people don't need anything close to pilot training when learning to ride a moped---even though some could argue they should.

Take a look at the default skill-listing and what categories each skill falls under; mental skills are typically \textit{Int}-based, physical skills are \textit{Agi}, \textit{Con}, or \textit{Str}-based, social relies on \textit{Cha}, and the technical ones are primarily \textit{Dex}-based.

Would it make sense to make a \textit{Wis}-based category? Animal Handling and other nature related skills comes to mind as an example.

\paragraph{Example Skills}
\begin{center}
\begin{tabular}{r|l}
    \textbf{Skill Name}      & \textbf{Traits} \\\hline
    \textit{Pyromancy}       & $Wis + Wis$\\
    \textit{Archery}         & $Agi + Dex$\\
    \textit{Animal Handling} & $Con + Wis$\\
    \textit{Riding}          & $Dex + Wis$\\
\end{tabular}
\end{center}

\subsection{Skills, Specialisations, or Exploits?}
The default skill listing is designed to work for most situations.
If something could reasonably exist as a narrower definition of a broader skill---\textit{Rabbit Care} under \textit{Animal Handling}, for example---then there isn't really any need to add a whole new skill to the listing to do this.

In the case your campaign has certain abilities or quirks that go beyond the standard skill-set, then that's essentially what exploits are for.
Exploits exist to enable per-character exceptions to the existing rules.

If however, you feel like too many players or characters end up having the same exploit, consider either making it a skill or adding it as a rule (Section~\ref{sec:hacking-rules}).