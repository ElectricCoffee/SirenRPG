\chapter{Defence \& Damage Resistance}\label{chap:defence}
\section{Introduction}
Let's face it, taking damage isn't fun, but unfortunately, it's a natural and almost unavoidable consequence of combat.
Luckily, there are ways to avoid, or at least mitigate some of the damage taken.
For this, there are three different things that might help you out: damage resistance, and active and passive defence.

% Todo:
% Rewrite "active" and "passtive" defence to be a single section about simply "defending", maybe later rope it in as a subsection under combat.
% keep going

\section{Defending}
Defence, is performed by your character during combat.
This refers to actions like dodging, blocking, or parrying an attack.
Note that blocking and parrying are treated as the same type of action, since blocking with a sword and parrying with a shield mechanically act the same way.
More often than not, defensive moves have a lower chance of success, but also have a very high reward, as they make it possible to completely avoid all damage from an attack.

There are two things a character can do when defending:
\begin{enumerate}
    \item \textbf{Dodge.} Roll against your \textit{Speed} trait $+ \mathit{DfM}$.
    \item \textbf{Parry.} Roll against half of your weapon/fighting skill, rounded down $+ \mathit{DfM}$.
\end{enumerate}

\paragraph{Note} As outlined in Section~\ref{sec:move-and-defend}, you get an additional $+2$ to your roll if you also choose to move out of the way of the attack, but doing so takes up your entire turn.

\subsection{Defence Modifier}
The Defence Modifier (DfM), is a modifier gained from your armour and equipment which is then added or subtracted from your defence move.
It is important to note, that the DfM only applies to your defensive actions, and does not offer any damage resistance. 
For that, please refer to Section~\ref{sec:damage-resistance}.

\paragraph{Example} Jean has been challenged by a rival clan's champion, who is charging at him, claymore raised high.
Jean's dodge is 6, his leather armour gives him $+1 \mathit{DfM}$ and his buckler gives him $+1 \mathit{DfM}$ to \textit{Parry}.

In order to maximise his defensive modifier, Jean decides to parry the attack while also side-stepping, giving him a total defence of 10.
Jean then rolls an 8, successfully parrying and avoiding any damage.
However, this ends Jean's turn.

\section{Damage Resistance}\label{sec:damage-resistance}
Damage resistance (DR), is something that always applies, whether your character is aware of the attack or not.
Damage resistance is granted by certain armour, shields, and potions, and reduce some of the damage your character takes when hit.
Simply put: if your character has 2 points of damage resistance, then they take 2 points of damage fewer than they otherwise would.

\paragraph{Example} Cass the Paladin is in a fight to the death against the fearsome black knight.
She wears a steel breastplate, which grants a damage resistance of 3.
The black knight swings at her and she misses her dodge!
The black knight's sword damages for 6, but because of her armour's damage resistance, she only takes 3 damage!
