\chapter{Game Creation}
\section{Telling a Story Together}
First and foremost, the point of playing an RPG is telling a story together, how you manage to go about doing so is completely up to you.
Creating a good story for your players to interact with can be a bit difficult at times, but it is absolutely doable.
Let your imagination run wild and see what you can come up with!
Generally speaking you need a few things:
\begin{itemize}
    \item A setting
    \item A scale
    \item A plot-line
    \item NPCs
\end{itemize}

\section{Setting}
The setting is one of the most important aspects of your game.
It determines everything from whether or not magic exists, to things like alien races, or level of technology.

Do you want to make 1940's style Noir story set in the distant future in a city that spans an entire planet? Go for it!
The most important thing, is that you---as the GM---can keep track of the setting, and keep it somewhat consistent.
Consistency is important so that things don't feel out of place. 
Don't be afraid to break consistency if you want to surprise or weird your players out.
It's your game world, make it however you want it.

\section{Scale}
When creating a game, it's also important to keep the scale of it all in mind.
Do you want a giant flourishing world filled to the brim with characters and opportunities on every corner? 
Or do you instead want something small and personal that can be wrapped up in a few sessions, but leaves a deep and profound impact on the players?
It's all up to you.
Just remember not to bite over more than you can chew, since bigger worlds tend to also require more work put into them.
It's easy to create this massive world and get lost in it.

\section{Plot}
What is a story without issues to resolve? Like any good movie or video-game, the story needs a plot-line.
Is a big evil corporation trying to take over the world?
Is a secret society conspiring against the general population?
Maybe a horde of dragons are stealing all the magic in the world for themselves in preparation for an upcoming war.

It's up to you, make it good.

\section{Non-Player Characters}
What is a world if there aren't anyone in it?
Just like the players need characters, so does the game world.
It's important to remember not only to make major plot-related NPCs, like a spy in the organisation you're trying to infiltrate, or the big bad guy that is foiling your plans at every step.
But minor NPCs like the slew of goons you want your players to fight, or that one one-off clerk at the shop that the players want to interact with.

NPCs generally come in those two flavours: Major and Minor.
Major NPCs are best created with a full character sheet, like the players, but minor NPCs can make by with just the most important stats written down.