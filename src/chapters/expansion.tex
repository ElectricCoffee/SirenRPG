\chapter{Hacking Siren}
\section{Introduction}
Over the course of this book, we have often been writing things like \textit{``It is up to the GM to decide that...''}, or \textit{``if this doesn't fit your setting, you're free to...''}, etc.
This chapter will serve as a guideline for how to expand the rules of Siren as an RPG engine, and perhaps give you some ideas.

Siren Core is written as generic as possible in order to hit as wide as target as possible in terms of genres.
Naturally this also means not everything in this book will necessarily fit your goals as a GM, and that's okay.
Everything in Siren is meant to be changed, expanded, or modified in some way to better suit the game you want to play.
If the \textit{Skill} listing doesn't fit your epic space adventure, then it can be changed.
If you think \textit{Faith} really needs to be a \textit{Trait}, then you can add it.
Only your imagination sets the limit!

\begin{note}
    This chapter will not go into how to add \textit{Exploits} or \textit{Spells}.
    Please refer to the relevant chapters for information on that.
\end{note}

\section{Skills}
The skill-listing is flexible, and is designed as such.
If you're writing a campaign set in the Wild West, you probably don't need any of the \textit{Vehicle} skills.
Playing in a setting that has several different incompatible magic systems? Then it may be a good idea to add a skill for each.

\subsection{Trait Pairs}
A thing to keep in mind when designing new skills is what pair of traits the skill should have.
It is easy to think that you can just slap any two traits onto a skill and call it a day, but the problem might be a bit subtler than that.
Each trait represents an approach to solving a problem, and every skill relies on one or more such approaches in tandem.

Does it need careful consideration? Then you probably want \textit{Int} in there.

Does it rely on common sense or intuition? Then \textit{Wis} is a good bet.
This is actually the primary reason why \textit{Vehicle (Land)} is \textit{Wis}-based and not \textit{Int}-based; most people don't need anything close to pilot training when learning to ride a moped---even though some could argue they should.

Take a look at the default skill-listing and what categories each skill falls under; mental skills are typically \textit{Int}-based, physical skills are \textit{Agi}, \textit{Con}, or \textit{Str}-based, social relies on \textit{Cha}, and the technical ones are primarily \textit{Dex}-based.

Would it make sense to make a \textit{Wis}-based category? Animal Handling and other nature related skills comes to mind as an example.

\paragraph{Example Skills}
\begin{center}
\begin{tabular}{r|l}
    \textbf{Skill Name}      & \textbf{Traits}\\\hline
    \textit{Pyromancy}       & $Int + Wis$\\
    \textit{Archery}         & $Agi + Dex$\\
    \textit{Animal Handling} & $Con + Wis$\\
    \textit{Riding}          & $Dex + Wis$\\
\end{tabular}
\end{center}

\section{Adding or Changing Traits}
Your setting may not be satisfied with just the seven basic traits included in the core rules.
It could be that you feel it would be better with an eighth trait like \textit{Faith} or \textit{Psionics} or something else entirely.
Maybe you instead prefer combining some of the traits into broader categories, like combining \textit{Int} and \textit{Wis} into a joined \textit{Mental} trait.
It's all doable, should you want to do so.
Just keep in mind: it's easier to add traits than to remove or combine them, since the latter will inevitably require you to rewrite many of the \textit{Skills} to fit the new listing.

\section{Adding New Rules}
A lot of more complicated or complex rules are intentionally left out from the core rules in order to keep things simple.
Stuff like encumbrance, splitting mental and physical health, or even making special rules for vehicle-based combat come to mind.

You can add all those things in if you really want to, but should you?
Part of the appeal of a simpler system is to not be overburdened with complicated rules, and other RPG systems already exist that cater to people who yearn for simulating every detail.

If you wish to write a module adding more such things to the base game, please feel free to do so! Just keep in mind that the main ethos of \textit{Siren} is that of ease of use.

\subsection{Example Additions}

\paragraph{Encumbrance} Your character can comfortably carry weight up to $Con + Str$ kilograms. 
Every kilogram beyond that will add $1FP$ for every 20 minutes of carrying.

\paragraph{Mental Health} Mental health is $Con + Wis$ and represents how much mental stress your character can suffer before negative effects start to show.
Mental Health functions the same way as physical health ($HP$), except it can only be damaged by mental attacks.
Mental attacks can be anything from psionic attacks, to mental abuse, to being in a stressful environment your character doesn't feel comfortable in.
In a mundane setting Mental Health can replace $MP$ as outlined in Section~\ref{sec:traits-mp}.

\paragraph{Holding Your Breath} Your character can comfortably hold their breath for $Agi + Con$ seconds.
Every second beyond that requires a \textit{Constitution Save} to succeed.
If the save fails, your character will immediately gasp out for air and suffer whatever consequences doing so may entail.

\section{Changing the Core Mechanics}
Sometimes the basic rules themselves don't fit what you really wish to achieve.
If you wish to shake up the number or type of dice used for the system, you can do so; just be prepared that you may need to re-balance or even change the \textit{Skill} system entirely.

Maybe you instead wish to change the system from rolling \textit{under} a given rating to \textit{over}---similar to how it's done in many other RPG systems.
You can do that of course, but given how fundamental this style of rolling is to how the engine is constructed, it may be a difficult task.

\subsection{Example Changes}
\paragraph{Hex-Grid} A rather simple change could be to use a hex-grid instead of a square one for combat.
Movement will be restricted to six directions instead of eight, and some care will have to be put into calculating distances.

\paragraph{Polyhedral Damage Dice} Maybe you want your great-axe to use a $d12$ as a damage die instead of $2d6$, or let the spear use a $d4$ instead of $d6-1$.

\paragraph{Action Points} Instead of having combat run in phases, it could instead use Action Points.
Each character has $AP$ equal to their speed, and may carry out any number of proactive or reactive actions as long as there's enough $AP$ to do it.