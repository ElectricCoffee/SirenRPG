\chapter{Magic} \label{chap:magic}
\section{Introduction}
Some campaigns are set in a fantastical world, full of weirdness, improbable events, and of course: magic.
It is said that any technology that is sufficiently advanced, is completely indistinguishable from magic.
For this reason, we define the term \textit{Magic} to be a bit looser than traditionally.
So, whether we're talking beam sword wielding space wizards, battle mages, or even mutated people that gain power from sea slug juices;
if it seems to give super natural powers, it all just falls under this category.

\paragraph{Note} Spells are anything that is channelled through your body, regardless of how.
Whether the powers were given to you, or you were born with them, doesn't matter.

\section{Spells}
In order to cast a spell, it needs to be known to the character using it.
That is to say, it must be available on the character sheet in some capacity---either in the spell section or perhaps even as an item, depending on how the given setting deals with this.
For all intents and purposes, spells are treated like normal skill checks.

In other words, casting a spell requires two things: A skill roll, and---if the setting calls for it---enough mana.
Like all other skills, spells have four different outcomes, but being magic in nature, they act a bit differently from regular skills:
\begin{itemize}
  \item \textbf{Critical Success.} The spell is performed at double strength, at no extra cost.
  \item \textbf{Success.} The spell succeeds as intended.
  \item \textbf{Fail.} The spell is not performed, but MP is still lost.
  \item \textbf{Critical Fail.} The spell backfires at the GM's or the setting's discretion.
\\This could be the spell having negative unforeseen consequences, and/or the loss of MP.
\end{itemize}

\paragraph{Note} It's up to the GM or the setting to determine the difficulty of a spell.
The difficulty would be set as a $\pm$ skill modifier on the given spell.

\subsection{Affecting the Spells}
Spells come in many different forms: utility, healing, hurting, elemental, bionic etc.
Every spell has an associated mana cost, which consumes a bit of your total mana on each use (see Section~\ref{sec:mana}).

It's up to the players and the GM to decide the kinds of spells that exist within the game, and whether they're too game breaking or not.
Spells can be cast in four different ways:
\begin{enumerate}
  \item On yourself.
  \item Around you.
  \item Away from you.
  \item On someone else.
\end{enumerate}
Depending on the exact nature of that spell, these four casting methods can have vastly different effects, or even be unavailable altogether.

\paragraph{Example} Casting a shield spell on yourself grants you protection.
Casting it around you creates a bubble around you that can shield yourself and your friends.
While casting it away from you can create a wall to shield something/someone further away.

\section{Mana (MP)}\label{sec:mana}
Every character has a mana pool $Int \times Wis$ in size, which represents the amount of magical potential your character can tap into without straining themselves.
In other words, if your character has 5 Int, and 6 Wis, they have 30 Mana Points (MP).
Every spell has an associated mana cost, and using it drains the mana pool accordingly.

\subsection{Running Out of Mana}
When you run out of mana, your can still cast spells!
However, instead of draining mana, your character will instead accumulate 2 Fatigue Points (FP) for every point of mana attempted to use.

\subsection{Regaining Mana}
Mana is regained automatically over time.
About one point every 30 in-game minutes while not at rest,
a point every 10 minutes while relaxing,
and a full restoration after a good night's sleep.

If the setting allows it, mana potions can also help restore some of the spent mana.

\section{Acquiring New Spells}
You can cast any spell your character would logically be acquainted with within the boundries of the setting.
It is, however, up to the GM to decide the number of spells you are allowed to start out with.
Spell familiarity is determined through the skill proficiency system, see Chapter~\ref{chap:skills} for more info.

By default however, the GM decides the XP cost of any given spell, or whether or not characters have \textit{spell slots} that limit the number of spells held. 
The actual intricacies of how a spell is learned is also up to the GM.

\paragraph{Note} In a setting, where it has been decided by the GM that a scroll needs to be found and read in order to learn a spell.
It is possible for the skill check to learn and/or cast the spell is too difficult for the character's current skill level.
This means, XP need to be dedicated to the given spell in order to properly learn it.