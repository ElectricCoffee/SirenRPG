\chapter{Magic} \label{chap:magic}
\section{Introduction}
Some campaigns are set in a fantastical world, full of weirdness, improbable events, and of course: magic.
It is said that any technology that is sufficiently advanced, is completely indistinguishable from magic.
For this reason, we define the term \textit{Magic} to be a bit looser than traditionally.
So, whether we're talking beam sword wielding space wizards, battle mages, or even mutated people that gain power from sea slug juices;
if it seems to give super natural powers, it all just falls under this category.

\section{Basics}
The magic system consists of two things: spells and mana.
Spells are the things your magic abilities can do, and mana is the stuff that fuels the spells.
No mana, no spells.

\paragraph{Note} Spells are anything that is channeled through your body, regardless of how.
Whether the powers were given to you, or you were born with them, doesn't matter.

\section{Spells}
Spells come in many different forms: utility, healing, hurting, elemental, bionic etc.
Every spell has an associated mana cost, which consumes a bit of your total mana on each use (see Section~\ref{sec:mana}).

It's up to the players and the GM to decide the kinds of spells that exist within the game, and whether they're too game breaking or not.
Examples of spells include:
\begin{center}
  \begin{tabular}{c|c|c|c}
    Healing & Fireball & Lightning Strike & Mind Control \\
    Illusion & Freeze & Overload & Stop Time \\
    Push & Raise Dead & Create Golem & Shield
  \end{tabular}
\end{center}
Spells can be cast in three different ways:
\begin{enumerate}
\item On yourself.
\item Around you.
\item Away from you.
\end{enumerate}
Depending on the exact nature of that spell, these three casting methods can have vastly different effects.

\paragraph{Example} Casting a shield spell on yourself grants you protection.
Casting it around you creates a bubble around you that can shield yourself and your friends.
While casting it away from you can create a wall to shield something/someone further away.

\section{Mana}\label{sec:mana}
Every character has a mana pool $Int \times Wis$ in size, which represents the amount of magical potential your character can tap into without straining themselves.
In other words, if your character has 5 Int, and 6 Wis, they have 30 Mana Points (MP).
Every spell has an associated mana cost, and using it drains the mana pool accordingly.

\subsection{Running Out of Mana}
When you run out of mana, your can still cast spells!
However, instead of draining mana, your character will instead accumulate 2 Fatigue Points (FP) for every point of mana attempted to use.

\section{Casting a Spell}
Casting a spell requires two things: A skill roll, and enough mana.
Like all other skills, spells have four different outcomes, but being magic in nature, they act a bit differently from regular skills:
\begin{itemize}
\item \textbf{Critical Success.} The spell is performed at double strength, at no extra cost.
\item \textbf{Success.} The spell succeeds as intended.
\item \textbf{Fail.} The spell is not performed, but MP is still lost.
\item \textbf{Critical Fail.} The spell backfires at the GM's discression. All MP is lost.
\end{itemize}

\paragraph{Note} It's up to the GM or the setting to determine the difficulty of a spell.
The difficulty would be set as a $\pm$ skill modifier on the given spell.

\section{Charging a Spell}
A character can charge a spell if they wish to increase its power.
You can charge a spell by multiplying the amount of mana used the cast it;
this also multiplies its potency.
Every level of charging requires a new dice roll to maintain concentration, if any of them fail, the whole spell fails.

During combat, charging a spell takes multiple rounds.
If you want to cast a spell that is three times more powerful than normal, casting it takes 3 whole rounds of charging.
You lose any and all charge if you get distracted (damaged, tripping, etc.).

\paragraph{Example} Harold wants to burn a troll he and his party are fighting.
His fire spell consumes 5 MP, and decides to charge it for three turns.
At the beginning of his third turn, Harold unleashes an enormous cone of fire from the tip of his wand, consuming 15 MP in the process.

\section{Regaining Mana}
Mana is regained automatically over time.
About one point every 30 in-game minutes while not at rest,
a point every 10 minutes while relaxing,
and a full restoration after a good night's sleep.

If the setting allows it, mana potions can also help restore some of the spent mana.

\section{Acquiring New Spells}
You can cast any spell your character would logically be acquainted with within the boundries of the setting.
It is, however, up to the GM to decide the number of spells you are allowed to start out with.
Spell familiarity is determined through the skill proficiency system, see chapter~\ref{chap:skills} for more info.

\paragraph{Note} Say your character finds a scroll with some mystical spell on it.
The scroll will give your character the knowledge of how to perform the spell, but they might still be unable to perform it well.
It will naturally take some time for your character to be familiar enough with the spell to perform it well.
Additionally, depending on the difficulty of the spell, it could be effectively uncastable at your character's skill level.
