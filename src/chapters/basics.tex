\chapter{Getting Started}
\section{What You Need}
\begin{enumerate}
    \item \textbf{Three to six players.} One of which will have to be the Game Master (GM).
    \item \textbf{Character sheets.} One per player. Additional paper is encouraged, but nor required.
    \item \textbf{GM sheet.} Special sheet for the Game Master.
    \item \textbf{Dice!} At least 3d6 per player, more is not necessarily required, but encouraged.
\end{enumerate}

\subsection{What's a Game Master?}
The Game Master is the special player in a game that directs the flow of the adventure.
The GM is responsible for narrating, setting up obstacles, and role-playing the Non-Player Characters (NPCs) that the players have to face throughout their adventures.

As a GM, the world is at your discretion. 
You roll the dice for everyone that isn't controlled by the players, you set the tone and mood of the environment, and you create the scenarios that will shape the decisions of the players.
You also get to set the difficulty of the obstacles at hand.

\subsection{The Role of the Player}
The player's role is to play their character, and face the trials imposed by the GM.
As a player, you control one of the protagonists in the story that the GM is telling, you make their choices and roll their dice to see how well they perform certain tasks.

\section{Character Sheet}

\section{Performing Actions}
Actions are what usually drives the plot in an RPG, whether it's talking to an NPC or performing great heroic feats, it's all actions.

Some actions have a chance of failure, and thus require you to roll your dice (typically 3d6) to see if you succeed or not. 
These are performed as \textit{Skill Rolls} and are covered more in depth in chapter \ref{chap:skills}. 
The success or failure of this roll decides the fate of your character.

Other actions succeed automatically, and do not require a skill roll, unless there's a very good reason that there should be.
Imagine having to roll your dice every time you wanted to talk to an NPC or walk down the street. 
Unless your character has crippling social anxiety or completely lacks a balance, it's assumed that these sorts of things are automatically successful.