\chapter{Traits}
\section{Introduction}
Traits are the aspects that define your character. 
These are things like their brains, their brawn, their fitness, etc. 
There are two different types of Traits in the game: \textit{Core Traits}, and \textit{Derived Traits}, which are special Traits derived from the core ones.
\section{Core Traits}
There are 7 core traits, each with an associated three-letter abbreviation.
Traits have an associated score, the score determines how good or bad your character will be at certain tasks during the game. 
You can find the core traits in Figure~\ref{fig:core_traits}.

\begin{figure}[!h]
    \centering
    \begin{tabular}{r | l}
        \textbf{Trait}  & \textbf{Description} \\\hline
        Agility      ($Agi$) & Movement and reaction \\
        Charisma     ($Cha$) & Ability to affect people \\
        Constitution ($Con$) & Health, stamina, and mettle \\
        Dexterity    ($Dex$) & Use of your hands and fingers \\
        Intelligence ($Int$) & Mental capability \\
        Strength     ($Str$) & Raw strength and muscle power \\
        Wisdom       ($Wis$) & Situational assessment and intuition
    \end{tabular}
    \caption{Core Traits}
    \label{fig:core_traits}
\end{figure}

\subsection{Agility (Agi)}
Agility represents your character's swiftness, manoeuvrability, and general fitness.
An agile character naturally lends themselves to speed, be it in terms of running or reaction time.

Agility is a core component in most physical skills; so whether your character is a professional boxer, an athlete, a bouncer, or a courier; Agility is the trait for you.

\subsection{Charisma (Cha)}
Charisma is your \textit{people skill}.
It represents your character's general ability to interact with, affect, and manipulate people.
A highly charismatic character can talk their way into or out of almost any situation, and as such Charisma is a core component in most of the social skills.

Whether your character is a sleazy con-artist, a performer, or a suave secret agent, Charisma is likely an important trait for you to consider.

\subsection{Constitution (Con)}
Constitution determines your character's health, their stamina, and their willpower, in other words: your character's general stubbornness.
Constitution is the core trait in everything to do with physical and mental health, and has a key role in a number of skills related to yours and other people's survival.

If your character is a medic, a ranger, or needs to tank a lot of hits, then Constitution is a good priority choice.

\subsection{Dexterity (Dex)}
Dexterity---or more accurately: \textit{manual} dexterity---governs your character's fine motor skills and ability to skilfully use their hands and fingers.
Intricate jobs like lock-picking, surgery, and watch-making all lend themselves nicely to a dexterous character, but coarser jobs that also require the use of your hands, like marksmanship and sailing also fall in this category.


\subsection{Intelligence (Int)}
Intelligence represents mental capacity, logical reasoning, and general academic prowess.
A highly intelligent character will have no trouble navigating academic resources and doing research; for this reason, most mental skills rely on Intelligence to some degree.

Good for characters who want to think their way out of and around problems.
Scholars, mages, and tacticians all benefit from a high Intelligence.

\subsection{Strength (Str)}
Strength represents the raw physical strength of your character.
Bulky and buff characters will naturally be strong, and are good for hitting hard and pulling the weight; this trait is key in some physical skills.

Bouncers, brutes, weight lifters, naturally benefit from a high Strength.

\subsection{Wisdom (Wis)}
Wisdom is your character's situational awareness, intuition, and general gut-feel.
For this reason, Wisdom applies to a wide variety of situations, as situational awareness and a good intuition will often be enough to get you through a lot of situations.

\section{Derived Traits}
The derived traits are all based on the core traits, and therefore cannot be chosen. 
You can find the derived traits on Figure~\ref{fig:derived_traits}.

\begin{figure}[!ht]
    \centering
\begin{tabular}{r | l}
    \textbf{Trait} & \textbf{Calculation} \\\hline
    Health  ($HP$) & $Con\times 2$ \\
    Fatigue ($FP$) & $Str + Con$ \\
    Mana    ($MP$) & $Int\times Wis$ \\
    Speed          & $Athletics / 2$\\
    Parry          & $Fighting/2 + \mathit{DfM}$\\
    Dodge          & $Speed + \mathit{DfM}$
\end{tabular}
    \caption{Derived Traits. (See Section~\ref{sec:defence} for $\mathit{DfM}$)}
    \label{fig:derived_traits}
\end{figure}

\subsection{Fatigue (FP)}
Fatigue represents how hard your character can exert themselves.
Every point of fatigue has adverse effects on your character's ability to perform.
Get too fatigued, and your character faints from exhaustion.
You can read more about Fatigue as a mechanic in Section~\ref{sec:fatigue-points}.

\subsection{Health (HP)}
Health represents a character's physical well-being.
If you have any past experience playing videogames or other RPGs, you know exactly what the purpose of Health is, but for those uninitiated, it is simply a measure of how close to dying your character is.
Worry not though, getting to zero HP does not instantly mean death!
More on that in Section~\ref{sec:health-points}.

\subsection{Mana (MP)}
Mana represents your character's magical potential and is used to cast spells (if the setting allows it).
Running out of mana runs the risk of exhausting your character, so use it effectively!
More about mana and how to use it in Section~\ref{sec:mana}.

\note If a certain setting calls for it, the MP can be used as ``Mental Points'' or ``Sanity Points'' instead of magical potential.
This is useful if you want to drain your players sanity over the course of campaigns that feature particularly horrifying things.
Though if you plan on using mana for that, consider calculating it as $Con \times Wis$ rather than $Int \times Wis$.

\subsection{Parry}
This derived trait is a bit different from the others, in that it's based on your character's \textit{currently active Fighting skill}.
In other words, you use \textit{Fighting (Light)} if unarmed or using a light weapon, and \textit{Fighting (Heavy)} if using a heavy weapon.
More on parrying in Section~\ref{sec:defence}.

\subsection{Speed}
If playing with a battle-mat, the \textit{Speed} trait determines how many squares a character can move per turn.
More on that in Section~\ref{sec:movement}.

\subsection{Dodge}
The Dodge trait determines your character's ability to dodge out of the way of danger. 
By default it is equal to your character's Speed trait + the appropriate \textit{DfM}.
More about Dodge in Section~\ref{sec:defence}.


\section{Trait Saves}
Sometimes when a situation calls for it, you need to roll what's called a \textit{Trait Save}.
Trait saves are where you roll against your trait rather than your skill to affect the outcome of a certain situation.

A trait save is always $2 \times \mathit{affected\ trait}$, regardless of the trait.
Trait saves are different from \textit{Skill Rolls} (see Chapter~\ref{chap:skills}) in that they only make use of a single trait, and are not subject to modifiers.

\example Mark is on the brink of collapse.
He's lost a lot of blood in the battle against the mob boss.
His \textit{Constitution} is at 6, and needs to roll a \textit{Constitution Save} ($Con \times 2 = 12$) to not faint from blood loss.
He rolls a 2, a 6, and a 5, a total of 13. He collapses and faints.
