\section{Healing Damage}
Healing usually takes place \textit{outside} of combat, but can also be performed during combat as a \textit{Long Action}.
Healing takes time, and depending on the severity of the wound, the healing process takes a different amount of time.

The skill used for healing entirely depends on the task at hand.
Surgeries, anæsthetics, medicine, therapy, etc. all fall under the \textit{Academics} skill, while caretaking falls under \textit{Nursing}.
If however, it's automatic, then it depends on the effectiveness of that particular remedy.
That is, applying bandages or stitching a wound depends on the skill of the doer, while drinking a magic potion is dependent on the strength of the potion.

Healing comes in three different variants:

\begin{center}
  \begin{itemize}
  \item \textbf{Healing by mending.}
    This is the slowest form of healing, refers to any kind of healing performed via bandages, stitches, rest, or similar.\\
    Mending happens over the course of days or weeks.
  \item \textbf{Healing by potion.}
    The second-fastest form of healing, which refers to injecting or ingesting a substance that has regenerative properties. \\
    Potions heal over the course of minutes or hours.
  \item \textbf{Healing by magic.}
    The fastest form of healing, which refers to any procedure that facilitates instant healing, magic or otherwise.
  \end{itemize}
\end{center}

\paragraph{Note} Healing refers to both healing Health and Fatigue.
A cup of coffee, for example, is a fatigue-healing potion that takes about 15 in-game minutes to kick in.
