\section{Phases}\label{sec:phases}
Turns in Siren is split into two distinct phases: \textit{Proactive} and \textit{Reactive}.

The phases are structured as follows:
\begin{enumerate}
  \item Proactive 
  \item Reactive
  \item Damage Calculation (if applicable)
\end{enumerate}

\subsection{Proactive Phase}
The proactive phase is what you would think of as ``your turn'', during your proactive phase you can pick any two of the following actions: \textit{Moving}, \textit{Bracing}, or \textit{Attacking}.
If however, your character is performing a non-combat action (such as picking a lock), this takes up the entire proactive phase.

\paragraph{Move}
Movement means your character can move up to half their speed-rating.

\paragraph{Brace}
Your character takes a defensive stance and gets $+2$ to any defence rolls this turn.

\paragraph{Attack}
You character gets to roll an attack.

If your character performs a ranged attack, you need to check the weapon's range.
The skill check to attack is done at $-3$ for every time the range of the weapon is exceeded.

\paragraph{Ranged Attack Example} Will the Ranger is out hunting for his party. He spots a deer in the distance. His bow has a range of 100 metres, the deer is 300 metres away. This exceeds the bow's range twice ($2\times -3$ penalty). Will must roll $Longbow-6$ to succeed.

\paragraph{Note} In the proactive phase, your character can pick the same option twice; that is, any combination of the two, or the same option twice.

\subsection{Reactive Phase}
When your character is being attacked, they enter the reactive phase.
During this phase, you can choose any \textbf{one} of either \textit{Moving}, \textit{Defending}, \textit{Counter-Attacking}

\paragraph{Move}
Movement means your character can move up to half their speed-rating.

\paragraph{Defend}
Your character is able to perform a defensive roll + any cumulative modifiers from their previous proactive phase.

This means if you didn't move or brace yourself in the previous turn, you don't get the relevant bonuses.

Additionally, wielding an item in each hand capable of parrying, allows you to parry once for each item.

\paragraph{Counter-Attack}
Instead of moving or defending, you may simply initiate an attack in response.