\section{Phases}\label{sec:phases}
Turns in Siren are split into two distinct phases: \textit{Proactive} and \textit{Reactive}.

Each turn is structured as follows:
\begin{enumerate}
  \item Proactive 
  \item Reactive
  \item Damage Calculation (if applicable)
\end{enumerate}

\subsection{Proactive Phase}
The proactive phase is what you would think of as ``your turn'', during your proactive phase you can pick any two of the following actions: \textit{Attacking}, \textit{Bracing}, or \textit{Moving}.
If however, your character is performing a non-combat action (such as picking a lock), this takes up the entire proactive phase.

\paragraph{Attack}
You character gets to roll an attack.

If your character performs a ranged attack, you need to check the weapon's range.
The skill check to attack is done at $-3$ for every time the range of the weapon is exceeded.

\paragraph{Brace}
Your character takes a defensive stance and gets $+2$ to any \textit{Dodge} or \textit{Parry} rolls this turn.

\paragraph{Move}
Your character can move up to half their \textit{Speed}.

\begin{tcolorbox}[title = Ranged Attack Example] 
  Will, the Ranger is out hunting for his party. 
  He spots a deer in the distance. 
  His bow has a range of 100 metres, the deer is 300 metres away. 
  This exceeds the bow's range twice ($2\times -3$ penalty). 
  Will must roll $Longbow-6$ to succeed.
\end{tcolorbox}

\begin{note} 
  In the proactive phase, you can choose to pick the same option twice. 
  This allows you to \textit{Attack}, \textit{Brace}, or \textit{Move} twice in the same turn if you wish to.
\end{note}

\subsection{Reactive Phase}
When your character is being attacked, they enter the reactive phase.
During this phase, you can choose up to \textbf{two} of either \textit{Counter}, \textit{Dodging}, or \textit{Parrying}.

You cannot select the same option again until a \textit{full round of combat} has concluded, unless otherwise specified.

\begin{note}
  You get a reactive phase even if your opponent \textit{misses} their attack!
\end{note}

\paragraph{Counter}
Instead of dodging or parrying, you may simply initiate an attack in response.

Countering does not ignore damage.

\paragraph{Dodge}
Your character gets to roll to dodge out of the way.
Bonuses are applied if they \textit{Braced} in their proactive phase.

A successful dodge ignores damage from an attack and moves you into an adjacent square of your choice.

\paragraph{Parry}
Your character gets to roll to parry.
Bonuses are applied if they \textit{Braced} in their proactive phase.

A successful parry ignores damage from an attack during this reactive phase.

Additionally, wielding something in each hand capable of parrying, allows you to perform a \textit{Parry} reaction once for each item.
This scales with the number of hands your character may have.