\chapter{Character Progression}\label{chap:char-prog}
\section{Introduction}
Character progression is important in any RPG.
As your character progresses throughout the adventure, they're naturally going to acquire new abilities, and hone existing their skills.

\section{Experience Points}
To let your characters progress, the game uses \textit{Experience Points} (XP).
XP are used for three things:
\begin{enumerate}
\item Buy proficiency bonuses.
\item Buy trait upgrades.
\item Buy new exploits.
\end{enumerate}
The XP cost for each varies depending on what's being bought.

\section{Buying Proficiency Bonuses and Trait Upgrades}
The price of buying a proficiency bonus or trait upgrade is the same:
Each level costs itself in points.
Put differently: going from level 1 to 2 costs 2 points, and going from 7 to 8 costs 8 points.
Naturally, this means that if you want to buy three levels all at once, you have to pay the price of each successive level.

\paragraph{Example} Joyce wants to increase her warrior's \textit{Wisdom}, it's currently at level 2, and she wants to bring it up to level 5.
To make this happen, she needs to pay 12 XP, because she also needs to buy levels 3 and 4, making it $3 + 4 + 5 = 12$ XP.

\section{Buying Exploits}
As exploits can have a profound impact on the way a character is played, these are naturally much more expensive.
Each exploits costs four times the number of exploits already present, or visualised graphically; the first exploit is free, the next costs 4 XP, the next 8, and so on:

\begin{center}
  \begin{tabular}{c|c}
    \textbf{Exploit} & \textbf{XP Cost}\\\hline
    1 & Free!\\
    2 & 4 \\
    3 & 8 \\
    4 & 12 \\
    5 & 16 \\
    6 & 20 \\
  \end{tabular}
\end{center}

\paragraph{Example} Joyce also wants to buy two new exploits for her warrior.
The warrior already has one exploit, and she needs to pay 12 XP to add those extra exploits, since exploit no.\ 2 costs 4, and no.\ 3 costs 8, the total comes up to $4 + 8 = 12$ XP.

\section{Getting XP}
Having an XP system wouldn't make much sense if it wasn't also possible to gain more of it.
\textit{Milestones} are used to accomplish this task.
A milestone is a significant moment in the game that creates a natural \textit{break} in the gameplay.
Examples of natural milestones could be completing a mission or a story arc, at the defeat of some villain that your party has been up against, or even simply at the end of a session.
It is entirely up to the GM to distribute XP, and to help with this, milestones are split into three different kinds: minor, significant, and major.

\paragraph{Note} XP is earned individually, not for the entire party.
It makes little sense that the entire party gets the same amount of XP, if Broot the Destroyer goes hunting orcs all by himself.

\subsection{Minor Milestone}
Minor milestones typically occur at the end of a session, or at the end of a minor in-game event.
Minor milestones should give you a chance to re-balance your character and re-word one of your exploits.

Your character gains 1 XP.

\subsection{Significant Milestones}
Significant milestones occur at important in-game events.
It could be that the party finally found the evil wizard's secret lair, or that some moderate challenge has been overcome.
If in doubt, this can also just happen every 2--3 sessions.

Your character gains 2 XP.

\subsection{Major Milestones}
Major milestones occur at events that shake things up a lot.
Things like, killing the main villain, or driving a group of bandits out of town, or perhaps wreaking a significant amount of havoc.
Alternatively, it could also be at the end of a story arc.

Your character gains 4 XP.

\paragraph{Note} It's of course up to the GM how much XP is awarded, this is meant as more of a guide than a hard rule.
