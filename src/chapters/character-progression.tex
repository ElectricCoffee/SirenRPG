\chapter{Character Progression}\label{chap:char-prog}
\section{Introduction}
Character progression is important in any RPG.
As your character progresses throughout the adventure, they're naturally going to acquire new abilities, and hone existing their skills.

\section{Experience Points}
To let your characters progress, the game uses \textit{Experience Points} (XP).
XP are used for three things:
\begin{enumerate}
\item Buy proficiency bonuses.
\item Buy trait upgrades.
\item Buy new exploits.
\end{enumerate}
The XP cost for each varies depending on what's being bought.\\
\textbf{XP is always spent during a long rest (a rest of 8+ hours).}

\section{Buying Trait Upgrades}
The price of buying a trait upgrade is always the same:
Each level costs just one (1) point of XP, \textbf{unless} the player wishes to upgrade the skill or trait more than once during a single long rest, in which case the price for a level increases by 1 XP for each additional level.

\paragraph{Example} This levelling session, Joyce wants to increase her character's \textit{Wisdom}. 
It's currently at level 2, and she wants to bring it up to level 3.
For this to happen, she pays 1 XP for \textit{Wisdom}.
She also wants to bring her level of \textit{Strength} from 3 to 5, that will cost her 1 XP from 3 to 4, and 2 XP from 4 to 5, or 3 XP total for \textit{Strength}.

\section{Buying or Upgrading Specialisations}
Every new specialisation or upgrade to such, always just costs 1 XP regardless of how many points you wish yo add during a levelling session.

\section{Buying Exploits}
As exploits can have a profound impact on the way a character is played, these thus have a chance of being more expensive than traits or specialisations.
It is generally up to the GM how much an exploit will cost, typically between 1 and 4 XP, though they can cost more.

Exploit costs can often be split into three tiers:

\begin{tabular}{r | l}
  1 XP & Exploits that have lots of restrictions or minor mechanical bonuses\\
  2 XP & Exploits that are more of a bonus with fewer restrictions\\
  3 XP & Exploits that change the nature of the character, with restrictions applied\\
  4 XP & Exploits that have few or no restrictions and/or have large mechanical effects.
\end{tabular}

\subsection{Tier 1 Examples:}
\paragraph{Bloodtinged (1 XP)}
When in melee combat, you deal $+2$ bonus damage when fighting non-human opponents.

\paragraph{I Need Some Duct-Tape! (1 XP)}
\textbf{Requirements:} At least two \textit{Crafting} specialisations and access to duct-tape.

When deprived of the correct tools, materials, and/or workspace you are still able to make crafting checks so long as you are able to give an explanation to the GM of how your character did so.
You do so with a $-3$ modifier to your \textit{Crafting} check.

\subsection{Tier 2 Examples:}
\paragraph{Bloodlust (2 XP)}
When in melee combat, a roll of 4 or lower counts as an automatic success against non-human opponents.

\paragraph{Give Me a Couple of Bobby-Pins (2 XP)}
\textbf{Requirements:} At least two \textit{Crafting} specialisations.

When deprived of the correct tools, materials, and/or workspace you are still able to make crafting checks so long as you are able to give an explanation to the GM of how your character did so.

\subsection{Tier 3 Examples:}
\paragraph{Blooddrunk (3 XP)}
When in melee combat, a roll of 4 or lower counts as an automatic success.

\paragraph{Mac's Back (3 XP)}
\textbf{Requirements:} At least one \textit{Crafting} specialisation.

When deprived of the correct tools, materials, and/or workspace you are still able to make crafting checks so long as you are able to give an explanation to the GM of how your character did so.
Additionally, the maximum penalty on any \textit{Crafting} check is now $-2$.

\subsection{Tier 4 Examples:}

\paragraph{Blood Frenzy (4 XP)}
When in melee combat, a roll of 5 or lower counts as an automatic success.
Additionally, you deal $+2$ bonus damage against non-human oppponents.

\paragraph{Swiss Army Man (4 XP)}
\textbf{Requirements:} none.

When deprived of the correct tools, materials, and/or workspace you are still able to make crafting checks so long as you are able to give an explanation to the GM of how your character did so.
Additionally, the maximum penalty on any \textit{Crafting} check is now $-2$.
\\\\See Appendix~\ref{app:exploits} for more pricing examples.

\section{Acquiring XP}
Having an XP system wouldn't make much sense if it wasn't also possible to gain more of it.
\textit{Milestones} are used to accomplish this task.
A milestone is a significant moment in the game that creates a natural \textit{break} in the gameplay.
Examples of natural milestones could be completing a mission or a story arc, at the defeat of some villain that your party has been up against, or even simply at the end of a session.
It is entirely up to the GM to distribute XP, and to help with this, milestones are split into three different kinds: minor, significant, and major.

\paragraph{Note} XP is earned individually, not for the entire party.
It makes little sense that the entire party gets the same amount of XP, if Broot the Destroyer goes hunting orcs all by himself.

\subsection{Minor Milestone}
Minor milestones typically occur at the end of a session, or at the end of a minor in-game event.
Minor milestones should give you a chance to re-balance your character and re-word one of your exploits.

Your character gains 1 XP.

\subsection{Significant Milestones}
Significant milestones occur at important in-game events.
It could be that the party finally found the evil wizard's secret lair, or that some moderate challenge has been overcome.
If in doubt, this can also just happen every 2--3 sessions.

Your character gains 2 XP.

\subsection{Major Milestones}
Major milestones occur at events that shake things up a lot.
Things like, killing the main villain, or driving a group of bandits out of town, or perhaps wreaking a significant amount of havoc.
Alternatively, it could also be at the end of a story arc.

Your character gains 4 XP.

\paragraph{Note} It's of course up to the GM how much XP is awarded, this is meant as more of a guide than a hard rule.
