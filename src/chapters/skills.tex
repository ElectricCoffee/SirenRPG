\chapter{Skills} \label{chap:skills}
\section{Introduction}
Skills are the primary workhorse in the game. 
Whenever there's any situation that requires action, a player rolls against their skill plus/minus modifier. 
There are 19 different skills, as seen in figure \ref{fig:skills}. 
These skills reflect the different types of actions a player can perform during play.


\section{Skill Rolls}
The player rolls three six-sided dice (3d6) against their skill of choice, if the outcome is $\leq$ the skill level, it's a success, if not, it's a fail.
Rolling a 3, is an automatic success (regardless of level), and rolling an 18 is an automatic fail (also regardless of level).

\paragraph{Example} Say Alice wants to persuade a guard to do something stupid. Her \textit{Persuasion} skill is at 13. 
She rolls 3d6, and rolls a 1, a 5, and a 3, which is 9 in total. This means she successfully persuades the guard.

\begin{figure}
\centering
\begin{tabular}{r | c | c | l}
\textbf{Category} & \textbf{Skill}   & \textbf{Traits} & \textbf{Description} \\\hline
    Mental    & Academics        & $Int+Int$ & Use vast knowledge of a certain field.        \\
              & Investigation    & $Int+Wis$ & Searching for things and information.         \\
              & Magic            & $Int+Wis$ & Perform fantastical feats.                    \\
              & Perception       & $Int+Wis$ & The passive ability to spot things.           \\
              & Survival         & $Con+Int$ & Surviving in certain environments.            \\
              & Thievery         & $Dex+Int$ & Performing certain larcenous activities.      \\\hline
    Physical  & Athletics        & $Agi+Con$ & Perform a certain task that requires stamina. \\
              & Fighting (Heavy) & $Agi+Str$ & The ability to fight with heavy weapons.      \\
              & Fighting (Light) & $Agi+Dex$ & The ability to fight with light weapons.      \\
              & Physique         & $Con+Str$ & Any action that requires strength.            \\
              & Stealth          & $Agi+Wis$ & Sneaking around and acting unseen.            \\\hline
    Social    & Contacts         & $Cha+Int$ & Make and use connections with people.         \\
              & Insight          & $Cha+Wis$ & Sensing and social cues and motivation        \\
              & Nursing          & $Cha+Wis$ & Taking care of people                         \\
              & Persuasion       & $Cha+Wis$ & Manipulating people.                          \\\hline
    Technical & Crafting         & $Dex+Int$ & Making things with your hands.                \\
              & Shooting         & $Dex+Wis$ & Shooting or throwing objects.                 \\
              & Vehicle (Air)    & $Dex+Wis$ & Operating motorised air vehicles.             \\
              & Vehicle (Land)   & $Dex+Wis$ & Operating motorised land vehicles.            \\
              & Vehicle (Sea)    & $Dex+Wis$ & Operating motorised sea vehicles.             \\
\end{tabular}
\caption{Skill Listing}
\label{fig:skills}
\end{figure}
\newpage
\section{Proficiency \& Specialisation}
Your character can specialise in any number of different skills.
Specialising grants the skill a \textit{Proficiency Bonus} of $+3$ to that particular specialisation.

\paragraph{Note} Just because your character hasn't specialised in anything, does not mean they cannot perform it, it just means they don't get the proficiency bonus.

\paragraph{Example} Joyce is an Olympic runner, and has a \textit{Constitution} of 5, and an \textit{Agility} of 5; this gives her an \textit{Athletics} score of 10. 
As a runner, she specialises in running, which grants her a $+3$ bonus to her \textit{Athletics} roll whenever she needs to perform a running task.
She rolls 11, but because her effective \textit{Athletics} score is 13 due to the bonus, she succeeds.

\newpage
\section{Skills in Depth}
\subsection{Mental Skills}
\subsubsection{Academics (Int + Int)}
The Academics skill, called \textit{Knowledge}, or \textit{Lore} in other games, is the skill of knowing things factually. 
This can be things your character has read, or something they've researched.
If your character needs to identify ancient writing, source herbs, design circuits, or have a meaningful discussion with an expert, this is the go-to skill.

\paragraph{Specialisations:}
\begin{center}
    \begin{tabular}{c|c|c|c}
        Maths & Physics & Chemistry & Theology \\
        Medicine & Literature & Linguistics & History \\
        Anthropology & Computer Science & Magic & Botany \\
        Geology & Astronomy & Quantum Physics & Computer Security \\
    \end{tabular}
\end{center}

\subsubsection{Investigation (Int + Wis)}
Investigation is the skill of actively searching for things and gathering information.
Whether your character is hunting for clues or digging through books, investigation is the skill to use.

\subsubsection{Magic (Int + Wis)}
Magic is the skill of performing fantastical things with power channeled through the body.
Whether it's chi energy, spiritual power, nanobots, genetic modification, or the essence of the universe that powers your abilities, it all falls under \textit{Magic}.

\paragraph{Specialisations:}
\begin{center}
  \begin{tabular}{c|c|c|c}
    Life  & Death & Construction & Destruction \\
    Earth & Water & Air & Fire \\
    Arcane & Black/Dark & White/Light & Temporal \\
    Divine & Diabolic   & Utility
  \end{tabular}
\end{center}
Magic is covered much more in depth in chapter \ref{chap:magic}.

\subsubsection{Perception (Int + Wis)}
Perception, called things like \textit{Notice} or \textit{Awareness} in other games, is the skill of using your senses to \textit{passively} notice what's going on around you.

\subsubsection{Survival (Con + Int)}
Survival is the skill of surviving; particularly in inhospitable environments.

\paragraph{Specialisations:}
\begin{center}
    \begin{tabular}{c|c|c|c}
        Desert & Tundra & Jungle & Island \\
        Prairie & Urban & Off-World & Space \\
        Under Water & Underground & Mountain & Arctic \\
    \end{tabular}
\end{center}

\subsubsection{Thievery (Dex + Int)}
Thievery is a catch-all term for general larcenous and/or criminal skills that (generally) involve thought and finesse.
These skills are regarded as such regardless of their actual use.

\paragraph{Specialisations:}
\begin{center}
    \begin{tabular}{c|c|c}
        Pick-Pocket & Lock-Pick & Sleight of Hand \\
        Alarm Deactivation & Hacking & Evidence Tampering \\
    \end{tabular}
\end{center}

\subsection{Physical Skills}
\subsubsection{Athletics (Agi + Con)}
Athletics, sometimes called \textit{Endurance} in other games, is a skill that has your character perform physically challenging things, often involving swiftness or stamina.
Tasks such as running, climbing, or swimming fall under here.

\paragraph{Specialisations:}
\begin{center}
    \begin{tabular}{c|c|c|c}
        Running & Climbing & Swimming & Jumping \\
        Acrobatics & Biking & Diving & Sailing \\
    \end{tabular}
\end{center}

\paragraph{Note} Sailing here refers to sailing a non-motorised sail or row boat. For motorised sailing, see the skill \textit{Driving}.

\subsubsection{Fighting (Heavy) (Agi + Str)}
This is the skill of melee fighting with heavy weaponry. Weapons such as clubs, mallets, war-hammers, axes, etc. fall under this skill.

\paragraph{Specialisations:}
\begin{center}
    \begin{tabular}{c|c|c|c}
        Mace & Club & Mallet & War Hammer \\
        Claymore & Baseball Bat & Pipe & Pipe Wrench \\
        Battle Axe & Two-by-Four & Odachi & Talwar \\
    \end{tabular}
\end{center}

\subsubsection{Fighting (Light) (Agi + Dex)}
This is the skill of melee fighting without weapons or wit light weaponry. 
Weapons such as rapiers, short swords, lances, and knives fall under here.

\paragraph{Specialisations:}
\begin{center}
    \begin{tabular}{c|c|c|c}
        Rapier & Short Sword & Katana & Beam Sword \\
        Knife & Ice Pick & Baton & Brass Knuckles \\
        Fist (punching) & Foot (kicking) & Rolling Pin & Letter Opener \\
    \end{tabular}
\end{center}

\subsubsection{Physique (Con + Str)}
Physique, the raw strength counterpart to Athletics. 
This skill involves everything that requires strength.
Lifting, pulling, or pushing heavy objects are things that fall under here.

\subsubsection{Stealth (Agi + Wis)}
Stealth is the act of moving around unnoticed.
Tasks such as blending in, sneaking, and infiltrating all fall under here.

\subsection{Social Skills}
\subsubsection{Contacts (Cha + Int)}
Contacts is the skill of creating, maintaining, and making use of contacts.
So if your character needs to call a friend for help, or establishing a new connection with someone for use later, this is the skill for you.

\subsubsection{Insight (Cha + Wis)}
Insight, called \textit{Sense Motive} or \textit{Empathy} in other games, is the dual/opposite of \textit{Persuasion}. 
This skill is all about reading people and their intentions, and trying to figure out what they're up to.
So if you're suspecting a salesman is trying to scam you, or if you're trying to figure out if an unstable inmate is about to have a wild mood swing, this is the go-to skill.

\subsubsection{Nursing (Cha + Wis)}
The Nursing skill is all about taking care of animals and people and making them feel good.
A good nurse can have a positive effect on someone's mental and physical health, and in fact can help speed up their natural healing.

\subsubsection{Persuasion (Cha + Wis)}
Persuasion is the skill for manipulating people for good or bad.
Whether you need to lure a guard away from his post, trick the evil sorcerer to reveal his secret plan, or strike a favourable deal with the mob boss, or maybe you just want to sit around in a pub cracking jokes, lightening the mood of the people around; this is the skill you want to use.

\paragraph{Specialisations:}
\begin{center}
    \begin{tabular}{c|c|c|c}
        Bargaining & Charming & Convincing & Inciting \\
        Seducing & Taunting & Provoking & Distracting \\
        Rapport & Deception & Bluffing & Intimidation
    \end{tabular}
\end{center}

\subsection{Technical Skills}
\subsubsection{Crafting (Dex + Int)}
The skill of making things, any hobby or task that involves producing something tangible with your hands belongs here.

\paragraph{Specialisations:}
\begin{center}
    \begin{tabular}{c|c|c|c}
        Paper Crafts & Origami & Carpentry & Woodworking \\
        Masonry & Smithing & Machining & Electronics \\
        Gadgeteering & Knitting & Casting & Sculpting \\
    \end{tabular}
\end{center}

\subsubsection{Vehicle, Air (Dex + Wis)}
The act of operating an airborne vehicle.

\paragraph{Specialisations:}
\begin{center}
    \begin{tabular}{c|c|c|c}
        Plane & Jet & Helicopter & Hot Air Balloon \\
        Commercial Liner & Zeppelin & Flying Car \\
    \end{tabular}
\end{center}

\subsubsection{Vehicle, Land (Dex + Wis)}
The act of operating land vehicles.

\paragraph{Specialisations:}
\begin{center}
    \begin{tabular}{c|c|c|c}
        Car & Truck & Tractor & Tank \\
        Train & Moped & Motorcycle & Hovercraft
    \end{tabular}
\end{center}

\subsubsection{Vehicle, Sea (Dex + Wis)}
The act of operating water based vehicles.
Note that sailing sail/row boats are Olympic disciplines, and thus fall under \textit{Athletics}.

\paragraph{Specialisations:}
\begin{center}
    \begin{tabular}{c|c|c|c}
        Jet-Ski & Motor Boat & Yacht & Cruise Liner \\
        Submarine & Hovercraft & 
    \end{tabular}
\end{center}

\subsubsection{Shooting (Dex+Wis)}
Shooting is the skill of launching projectiles, whether it be by throwing, or launching them from a device.

\paragraph{Specialisations:}
\begin{center}
    \begin{tabular}{c|c|c|c}
        Throwing & Slingshot & Bow & Crossbow \\
        Pistol & Revolver & SMG & Rifle \\
        Assault Rifle & Cannon & Ballista & Trebuchet \\
        Mortar & Tank Turret & Helicopter Turret & Blowpipe
    \end{tabular}
\end{center}

\newpage
\section{Contest of Skills}\label{sec:contest}
Situations arise, when your character needs to test their skills and prove their worth, this is done via a \textit{Contest of Skills}.
These contests come in three different variants: \textit{Passive}, \textit{Active}, and \textit{Tournament}.

\subsection{Passive Contest}
A passive contest arises when the character needs to overcome some passive obstacle. This could be things like walking a tightrope, picking a lock, leaping over a gap, etc.
The GM silently decides the level of difficulty by announcing a skill +/- a modifier that needs to be rolled.

\paragraph{Example} Astrid is running from her pursuers across the rooftops of ancient Rome, she needs to clear a particularly wide gap between the buildings. Astrid says \textit{``I'm jumping the gap''}, to which the GM replies \textit{``Roll Athletics -2''}. Astrid's Athletics skill is at 11, at -2, she must roll 9 or less to succeed. She rolls an 8 and gracefully leaps between the buildings.

\subsection{Active Contest}
An active contest is when the character is up against someone or something that actively works against them. Be it trying to pick-pocket a guard, an intense sword-fight, or playing chess against an opponent; all of this providing an \textit{active} resistance.

In an active contest, both parties roll against their most applicable skill, the outcome is judged as follows:

\begin{center}
    \begin{tabular}{c|c|c}
    \textbf{Character A} & \textbf{Character B} & \textbf{Winner}\\\hline
    Fail & Fail & The one failing by least \\
    Fail & Success & Character B \\
    Success & Fail & Character A \\
    Success & Success & The one succeeding by most
    \end{tabular}
\end{center}
If no one fails or succeeds more than the opponent, re-roll.

\paragraph{Example} Tom wants to sneak past a guard, Tom's \textit{Stealth} is at 9, he rolls 11, failing by 2 degrees. 
The guard's \textit{Perception} is at 10, but rolls 13, failing by 3 degrees. 
Tom failed less than the guard, thus successfully sneaking past the guard.

\subsection{Tournaments}
Tournaments are \textit{longer} contests of skills.
Tournaments are divided into rounds, whoever wins the most rounds wins the tournament.
Each individual round is a regular \textit{active contest}.
