\chapter{Exploits} \label{chap:exploits}
\section{Introduction}
An RPG is no fun if all the characters in a party are the same. 
\textit{Exploits} are a way for you to personalise your character. 
Make the character truly yours.

An exploit is simply anything that your character can do, given their background. In other games, these are typically referred to as \textit{Stunts}, \textit{Perks}, \textit{Feats}, or \textit{Class Abilities}.
Exploits however, are free-form in nature and do not lock your character into any particular path.

\begin{example}
    Billie is an intelligence agent, so naturally she has extensive training. 
    Because of this, she has the \textit{Just Your Friendly Neighbourhood Gardener} exploit, which gives her +2 to any \textit{Deception} roll while in disguise.
\end{example}

\section{Defining Exploits}

Exploits are something you---with the help of the GM---construct yourself. 
They are not necessarily defined anywhere, so it takes a bit of creativity to come up with something good.
In the case that there are pre-defined exploits, feel free to attempt to modify them to suit your character better.

Exploits are a way for you to define special skills for your character that fit their background in some way. 
Your character might have special training, great talent, a hobby, genetic modification, magical capability, etc. which would give your character a reason to be able to do that particular exploit.

Now, unlike regular skills---which everyone can do---it is important to note that exploits are special and individualise your character.
You can find a more comprehensive list of example exploits in Appendix~\ref{app:exploits}.

\subsection{Changing a Skill}
Changing a skill involves somehow modifying a skill's traits to suit your agenda. 
It is important to note that you are only allowed to modify the skill where it makes sense. 
\paragraph{Skill-Changing Examples}
\begin{itemize}
    \item \textbf{Intimidating Size.} You can use $Cha+Str$ instead of $Cha+Wis$ to intimidate someone.
    \item \textbf{Desert Dweller.} You can instinctively search for food in a desert. You can use $Con+Wis$ instead of $Con+Int$ when looking for food with \textit{Desert Survival}\footnote{Survival skill specialised for deserts}.
\end{itemize}

\subsection{Extending a Skill}
Extending a skill involves adding new actions that lets it do things it cannot normally do.

\paragraph{Skill Extending Examples}
\begin{itemize}
    \item \textbf{Machine Specialist.} You can use your \textit{Crafting} skill to deduce how any machine works regardless of prior familiarity.
    \item \textbf{Backstab.} You can attack someone with your \textit{Stealth} skill, provided they are not already aware of your presence.
    \item \textbf{Body Builder.} You can use your \textit{Physique} skill to distract people with your muscles.
\end{itemize}

\subsection{Empowering a Skill}
Another way of defining an exploit, is by giving a skill a bonus under some narrow circumstance.

\paragraph{Skill Empowering Examples}
\begin{itemize}
    \item \textbf{Master Strategist.} Get a $+2$ bonus to your \textit{Academics} roll when drawing up a war strategy.
\end{itemize}

\subsection{Changing the Odds}
Given a very narrow circumstance, you may choose to re-roll the least favourable die in your skill-roll.
\paragraph{Examples}
\begin{itemize}
    \item \textbf{De-Trapper.} Once per day, you may re-roll your least favourable die when disarming traps.
    \item \textbf{Lucky Dodge.} Once per day, you may re-roll your least favourable die when attempting to dodge.
\end{itemize}

\subsection{Free-Form Exploits}
Exploits do not necessarily have to fall within one of the aforementioned templates, and can be entirely free-form in their creation.

The free-form nature of these makes them naturally hard to balance, so discuss with your GM what the restrictions on the exploit should be.

\paragraph{Free-Form Examples}
\begin{itemize}
    \item \textbf{Mac is Back.} Requires at least one \textit{Crafting} specialisation.
    When deprived of the correct tools, materials, and/or workspace you are still able to make crafting checks so long as you are able to give an explanation to the GM of how your character did so.
Additionally, the maximum penalty on any \textit{Crafting} check is now $-2$.
\end{itemize}

These free-form exploits are useful for late-game characters that need to go beyond the usual limits of the game. A good way to balance these, would be to add prerequisite dependencies on other exploits or skills.

\section{Balancing Exploits}\label{sec:exploit-balance}
Exploits can vary quite a lot in terms of cost and power, this is intentional. 
A given GM may feel that a different cost for a given tier of Exploit better fits their style or setting.
A campaign about super-heroes would have exploits that would be unreasonable in a Noire Detective setting.
For this reason, there should have some checks and balances in place to ensure the exploits are not completely game-breaking or out of place.

Here is a non-exhaustive list of ways to balance exploits:
\begin{itemize}
    \item The exploit can only be used within a very narrow scope.
    \item The exploit has limited use-case.
    \begin{itemize}
        \item Can only be used $n$ times per in-game time-frame.
        \item Can only be used $n$ times per session.
        \item Has a limited number of uses that never regenerate.
    \end{itemize}
    \item The exploit has a contextualised cost. Examples include:
    \begin{itemize}
        \item Picking a lock, which always succeeds, but the lock breaks.
        \item Temporary negative modifiers.
        \item Lowers your character's social status.
    \end{itemize}
    \item As something becomes more guaranteed to work, the time it takes to do it also increases.
    \item Requiring lore/good explanation.
\end{itemize}

Some exploits are better suited as spells. 
Even if not working in a magical environment, having an associated spell cost could be beneficial.

\paragraph{Balancing Examples}
\begin{itemize}
    \item \textbf{Red Tape Splicer.} Requires \textit{Red Tape Recorder}.\\
    Unconditional re-roll on \textit{Persuasion} against a government official.\\
    Doing so damages your reputation with that agency.
\end{itemize}
See Appendix~\ref{app:exploits} for more example exploits.