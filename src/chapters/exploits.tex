\chapter{Exploits} \label{chap:exploits}
\section{Introduction}
An RPG is no fun if all the characters in a party are the same. \textit{Exploits} are a way for you to personalise your character. Make the character truly yours.

The dictionary defines an exploit to be \textit{``a striking or notable deed; feat; spirited or heroic act''}\footnote{Source: http://www.dictionary.com/browse/exploit}. 
Though for the purposes of the game we're going with a slightly more loose definition: it's simply anything that your character can do, given their background.

\paragraph{Example} Billie is an intelligence agent, so naturally she has extensive training. Because of this, she has the \textit{Just Your Friendly Neighbourhood Gardener} exploit, which gives her +2 to any \textit{Deception} roll while in disguise.

\section{Defining Exploits}

Exploits are something you---with the help of the GM---define yourself. 
They are not defined anywhere, so it takes a bit of creativity to come up with something good.

Exploits are a way for you to define special skills for your character that fit their background in some way. Your character might have special training, great talent, a hobby, genetic modification, magical capability, etc. which would give your character a reason to be able to do that particular exploit.

Now, unlike regular skills---which everyone can do---it's important to note that exploits are special and individual to your character.
You can find a more comprehensive list of example exploits in Appendix~\ref{app:exploits}.

\subsection{Changing a Skill}
Changing a skill involves somehow modifying a skill's traits to suit your agenda. It's important to note that you're only allowed to modify the skill where it makes sense. 
\paragraph{Examples}
\begin{itemize}
    \item \textbf{Intimidating Size.} You can use $Cha+Str$ instead of $Cha+Wis$ to intimidate someone.
    \item \textbf{Desert Dweller.} You can instinctively search for food in a desert. You can use $Con+Wis$ instead of $Con+Int$ when looking for food with \textit{Desert Survival}\footnote{Survival skill specialised for deserts}.
\end{itemize}

\subsection{Extending a Skill}
Extending a skill involves adding new actions that lets it do things it cannot normally do.

\paragraph{Examples}
\begin{itemize}
    \item \textbf{Machine Specialist.} You can use your \textit{Crafting} skill to deduce how a machine works by studying it.
    \item \textbf{Backstab.} You can attack someone with your \textit{Stealth} skill, provided they aren't already aware of your presence.
    \item \textbf{Body Builder.} You can use your \textit{Physique} skill to distract people with your muscles.
\end{itemize}

\subsection{Empowering a Skill}
Another way of defining an exploit, is by giving a skill a bonus under some narrow circumstance.

\paragraph{Examples}
\begin{itemize}
\item \textbf{Master Strategist.} Get a $+2$ bonus to your \textit{Academics} roll when drawing up a war strategy.
\end{itemize}

\subsection{Changing the Odds}
Given a very narrow circumstance, you may choose to reroll the least favourable die in your skill-roll.
\begin{itemize}
\item \textbf{De-Trapper} Once per game, you may re-roll your least favourable die when disarming traps.
\end{itemize}
